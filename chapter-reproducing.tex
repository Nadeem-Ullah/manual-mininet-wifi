\chapter{Reproducible Research}

The code and instructions used in this section can be found at \url{https://github.com/ramonfontes/reproducible-research}

\section{SwitchOn 2015}
\textit{Extended abstract - Towards an Emulator for Software Defined Wireless Networks}\\
\textbf{}
\\
Firstly, you have to get the code at \url{https://github.com/ramonfontes/reproducible-research/blob/master/mininet-wifi/SWITCHON-2015}, then execute it:

\begin{minted}{bash}
sudo python allWirelessNetworksAroundUs.py
\end{minted}

\begin{minted}{bash}
mininet-wifi> xterm sta1 h1
\end{minted}

\noindent On sta1:
\begin{minted}[breaklines]{bash}
 cvlc -vvv v4l2:///dev/video0 --input-slave=alsa://hw:1,0 --mtu 1000 --sout'#transcode{vcodec=mp4v,vb=800,scale=1,acodec=mpga,ab=128,channels=1}: duplicate{dst=display,dst=rtp{sdp=rtsp://10.0.0.10:8080/helmet.sdp}'
\end{minted}

\noindent On h1:
\begin{minted}[breaklines]{bash}
cvlc rtsp://10.0.0.10:8080/helmet.sdp
\end{minted}

\section{SBRC 2016}
\textit{DEMO - Mininet-WiFi: Emulação de Redes Sem Fio Definidas por Software com suporte a Mobilidade}

Codes necessary to reproduce this paper are available at \url{https://github.com/ramonfontes/reproducible-research/tree/master/mininet-wifi/SBRC-2016}

\subsection{Case 1 - Simple test}
\begin{minted}[breaklines]{bash}
sudo mn --wifi  
mininet-wifi>sta1 ping sta2  
mininet-wifi>sta1 iwconfig  
mininet-wifi>sta2 iwconfig
\end{minted}
   
\subsection{Case 2 (1/2) - Communication among stations and hosts/Verifying flow table}
\begin{minted}[breaklines]{bash}
sudo python examples/wifiStationsAndHosts.py  
mininet-wifi>nodes  
mininet-wifi>sh ovs-ofctl dump-flows ap1  
mininet-wifi>sta1 ping h3  
mininet-wifi>sh ovs-ofctl dump-flows ap1 
\end{minted}

\subsection{Case 2 (2/2) - Changing the controller (from reference to external controller)}

Open the code examples/wifiStationsAndHosts.py and make the following changes:

\begin{minted}[breaklines]{bash}
from: net = Mininet( controller=Controller, link=TCLink, switch=OVSKernelSwitch )  
to: net = Mininet( controller=RemoteController, link=TCLink, switch=OVSKernelSwitch )  
from: c0 = net.addController('c0', controller=Controller, ip='127.0.0.1' )  
to: c0 = net.addController('c0', controller=RemoteController, ip='127.0.0.1' )  
sudo python examples/wifiStationsAndHosts.py  
mininet-wifi>sta1 ping h3  #Why there is no communication? 
\end{minted}
  
\subsection{Case 3 - Handover}
\begin{minted}[breaklines]{bash}
sudo python examples/handover.py  
mininet-wifi>sta1 iwconfig  
mininet-wifi>sta1 ping sta2  #here I suggest you wait sta1 reaches ap2 before going to the next step
mininet-wifi>sta1 iwconfig  
\end{minted}
 
\subsection{Case 4 - Changes at runtime}
\begin{minted}[breaklines]{bash}
sudo python examples/wifiPosition.py  
mininet-wifi>sta1 iwconfig   
mininet-wifi>sta1 ping sta2  
mininet-wifi>py sta1.moveNodeTo('70,40,0')  
mininet-wifi>sta1 iwconfig  
mininet-wifi>sta1 ping sta2  
mininet-wifi>py ap1.setRange(60)  
mininet-wifi>sta1 iwconfig  
mininet-wifi>sta1 ping sta2  
\end{minted}

\subsection{Case 5 - Bridging physical and virtual emulated environments}

\begin{minted}[breaklines]{bash}
sudo systemctl stop network-manager
\end{minted}

Edit sbrc.py:

\begin{minted}[breaklines]{bash}
from: phyap1 = net.addPhysicalBaseStation( 'phyap1', ssid= 'SBRC16-MininetWiFi', mode= 'g', channel= '1', position='50,115,0', wlan='wlan11' )  
to: wlan11 to your usb wlan interface.
\end{minted}

Then,

\begin{minted}[breaklines]{bash}
sudo python sbrc.py
\end{minted}

At this moment users attending the conference will be invited to connect their mobile devices into the physical/emulated environment.

\section{SIGCOMM 2016}{\label{sec:sigcomm2016}}
\textit{Demo: Mininet-WiFi: A Platform for Hybrid Physical-Virtual Software-Defined Wireless Networking Research}

Requirements to reproduce:
\begin{itemize}
\item WiFi interface + (other WiFi or ethernet interface)
\item Floodlight OpenFlow controller
\item ofsoftswitch13 (you may install it with \textit{util/install.sh -3f}) - https://github.com/CPqD/ofsoftswitch13
\item Speedtest-cli
\item Codes are available at \url{https://github.com/ramonfontes/reproducible-research/tree/master/mininet-wifi/SIGCOMM-2016}
\end{itemize}

Important (changes in code - You have to set both Internet and wlan interfaces):
\begin{minted}[breaklines]{bash}
internetIface = 'eth0' # wired/wireless card.
usbDongleIface = 'wlan0' # wifi interface.
\end{minted}

\noindent Next, executing the Floodlight OpenFlow controller: 
\begin{minted}[breaklines]{bash}
sudo java -jar target/floodlight.jar
\end{minted}

\noindent Then,
\begin{minted}[breaklines]{bash}
   sudo py hybridVirtualPhysical.py
   mininet-wifi> sh ./rule.hybridVirtualPhysical
\end{minted}

Despite the content of \textit{rule.hybridVirtualPhysical} is included in \textit{hybridVirtualPhysical.py}, we have faced some troubles in floodlight controller. Thus, probably you have to execute rule.hybridVirtualPhysical after executing hybridVirtualPhysical.py.

Now, stations should be able to communicate with each other and with the Internet. You may use any station connected to any Access Point and try it out:
\begin{minted}[breaklines]{bash}
   mininet-wifi>xterm $station 
   $station>speedtest-cli
\end{minted}
Using speedtest-cli you can test both Download and Upload speed of your Internet connection. The available bandwidth is controlled by OpenFlow meter entries.\\

\noindent There is a web server accessible at 10.0.0.111 and according rules applied in rule.hybridVirtualPhysical, if you access 10.0.0.109 the traffic will be redirect to 10.0.0.111.\\
\\
\\
\textbf{Useful commands:}
\begin{minted}[breaklines]{bash}
sta1 iw dev sta1-wlan0 mpath dump #verify mesh routing information
sh dpctl unix:/tmp/ap3 stats-flow
sh dpctl unix:/tmp/ap3 stats-meter
sh dpctl unix:/tmp/ap3 meter-config
\end{minted}

\section{From Theory to Experimental Evaluation: Resource Management in Software-Defined Vehicular Networks}

Firstly, you have to get vanet.py at \url{https://github.com/ramonfontes/reproducible-research/blob/master/mininet-wifi/IEEE-Access-2017/vanet.py}

\begin{minted}[breaklines]{bash}
   sudo python vanet.py
\end{minted}

\noindent Please consider to watch: https://www.youtube.com/watch?v=kO3O9EwrP\_s

\section{The Computer Journal - How far can we go? Towards
Realistic Software-Defined Wireless
Networking Experiments}

Codes necessary to reproduce this paper are available at \url{https://github.com/ramonfontes/reproducible-research/tree/master/mininet-wifi/The-Computer-Journal-2017}

\subsection{\textit{Wireless n-Casting}}
\textbf{Requirements:\\}
\texttt{Floodlight} controller.
\\
\\
In order to reproduce this case, please follow the instructions below:
\\
\\
First, run the controller:
\begin{minted}[breaklines]{bash}
sudo java -jar target/floodlight.jar
\end{minted}


\noindent start the topology:
\begin{minted}[breaklines]{bash}
sudo python ncasting.py
\end{minted}

\noindent and finally install the rules:
\begin{minted}[breaklines]{bash}
sudo python ncasting-controller.py
\end{minted}

\subsection{\textit{Multipath TCP}}
\textbf{Requirements:\\}
You have to install \texttt{mptcp} and \texttt{ifstat} to reproduce this use case. \\
\\
In order to allow the communication we use pox controller with spanning tree enabled. This command can be used to enable spanning tree:
\begin{minted}[breaklines]{bash}
./pox.py forwarding.l2_learning openflow.spanning_tree --hold-down log.level --DEBUG samples.pretty_log openflow.discovery host_tracker info.packet_dump
\end{minted}

\noindent Then, start the environment in a new terminal and run some commands with xterm:
\begin{minted}[breaklines]{bash}
sudo python mptcp.py   
mininet-wifi>xterm sta1 sta1 h10 h10  
Node: h10 (terminal1)$ifstat
Node: sta1 (terminal1)$ifstat
Node: h10 (terminal2)$iperf -s
Node: sta1 (terminal2)$iperf -c 192.168.1.254 
\end{minted}

\subsection{\textit{Hybrid Physical-Virtual Environment}}
See Section~\ref{sec:sigcomm2016}.

\subsection{\textit{SSID-based Flow Abstraction}}
\textbf{Requirements:\\}
ofsoftswitch13
\\
\\
In order reproduce this case you have to run the following code:
\begin{minted}[breaklines]{bash}
sudo python forwardingBySSID.py
\end{minted}

\noindent Then, you may run any application (e.g., Iperf) to test the available bandwidth for any SSID. Alternatively, you might run dpctl to verify meter table configuration.
\begin{minted}[breaklines]{bash}
mininet-wifi> sh dpctl unix:/tmp/ap1 meter-config
\end{minted}

\subsection{\textit{EXPERIMENTAL VALIDATION: Propagation Model}}
Consider to use the file propagationModelCase.py if you want to reproduce the results. 

\subsection{\textit{EXPERIMENTAL VALIDATION: Simple File Transfer}}
Consider to use the files \textit{fileTransferring.py} and \textit{fileTransferring.cc} if you want to reproduce the results. 

\subsection{\textit{EXPERIMENTAL VALIDATION: Replaying Network Conditions}}
Consider to use files into the directory replayingNetwork/ if you want to reproduce the results.